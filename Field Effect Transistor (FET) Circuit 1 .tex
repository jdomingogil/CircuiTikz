% Field Effect Transistor (FET) - Circuit 1
% En el circuito de la figura, analice el comportamiento para las configuraciones de las variables de entrada (A=0, B=0, C=0) y (A=1, B=1, C=1) justificando el funcionamiento de los transistores. Vcc = 10 V

%In the circuit shown in the figure, analyze the behavior for the input variable configurations (A=0, B=0, C=0) and (A=1, B=1, C=1), justifying the operation of the transistors. Vcc = 10 V
% Author: José Domingo Gil Iglesias
% City-Country: Mérida (Spain)
% Date: April, 2025

\documentclass[border=10pt]{standalone}
\usepackage[american, siunitx]{circuitikz}

\begin{document}

\begin{circuitikz}
    \ctikzset{tripoles/mos style/arrows}
    \ctikzset{transistors/arrow pos=end}    
    
    \draw (0,0) node[nmos] (na) {\(N_A\)}
    (na.G) to[short, -o] ++(-1,0) node[left] {\(A\)}
    (na.D) --++(2,0) coordinate(a) to[short, -o]++(2,0) node[right] {\(f(A,B,C)\)}
    (0,-1.5) node[nmos] (nb) {\(N_B\)}
    (nb.G) to[short, -o] ++(-1,0) node[left] {\(B\)}
    (2,-0.75) node[nmos, xscale=-1] (nc) {} node[left] {\(N_C\)}
    (nc.G) to[short, -o] ++(1,0) node[right] {\(C\)}
    (nb.S) --++(1,0) node[ground] {} --++(1,0) -- (nc.S)
    (a) -- (nc.D)
    (0,1.5) node[nmos] (nr) {\(N_R\)}
    (nr.G) -|++(-0.5,1) coordinate(b) -- (b-|nr.D) -- (nr.D)
    (nr.D) to[short, -o] ++(0,1) node[above] {\(V_{cc}\)};    
\end{circuitikz}

\end{document}
